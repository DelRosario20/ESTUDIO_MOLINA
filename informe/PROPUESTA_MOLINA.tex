\documentclass[onecolumn,12pt]{article} 
\usepackage{times,xcolor}
\usepackage[utf8]{inputenc}
\usepackage[spanish]{babel}
\usepackage[hidelinks]{hyperref}
\usepackage{float}
\usepackage{natbib}
\usepackage{fancyhdr}
\usepackage{lastpage}
\usepackage{amsmath}
\usepackage{booktabs}
\usepackage{adjustbox}
\usepackage{caption}
\usepackage{placeins}
\usepackage{geometry}
\geometry{
    a4paper,
    left=2.5cm,
    right=2.5cm,
    top=2.5cm,
    bottom=2.5cm
}

\title{\textbf{PROPUESTA DE ESTUDIO}\\
\vspace{0.5cm}
\Large{Cointegración entre Remesas y Consumo de los Hogares en Ecuador: Evidencia Empírica del Período 2015-2024}}

\author{\\
Estudiante de Economía - UNEMI\\
\textit{Econometría Aplicada}}

\date{Noviembre 2025}

\begin{document}

\maketitle

\section{Tema y Contexto del Estudio}

\subsection{Tema de Investigación}
El presente estudio propone analizar la relación de equilibrio de largo plazo entre las remesas internacionales y el consumo de los hogares en Ecuador durante el período 2015-2024, mediante técnicas de cointegración. Esta investigación busca determinar si existe un vínculo estable entre estos flujos financieros externos y el gasto de las familias ecuatorianas.

\subsection{Contexto y Justificación}
Ecuador se posiciona como uno de los principales receptores de remesas en América Latina, con flujos que representan aproximadamente el 3-4\% del PIB nacional. La migración ecuatoriana, intensificada desde la crisis financiera de 1999, ha generado un flujo constante de remesas que constituyen una fuente fundamental de ingresos para miles de hogares.

La teoría económica del ingreso permanente de Friedman (1957) y la hipótesis del ciclo vital de Modigliani sugieren que las remesas, al ser percibidas como un ingreso relativamente estable, deberían tener un impacto significativo en el consumo de los hogares receptores. Este efecto opera a través de: (i) relajación de restricciones de liquidez, (ii) suavizamiento del consumo intertemporal, y (iii) efectos multiplicadores en la economía local.

\section{Marco Referencial Empírico}

\subsection{Artículo Principal de Referencia}

\textbf{Artículo Ancla:} Adams Jr, R. H., \& Cuecuecha, A. (2013). "The impact of remittances on investment and poverty in Ghana". \textit{World Development}, 50, 24-40.\\
\textbf{DOI:} \url{https://doi.org/10.1016/j.worlddev.2013.04.009}

Este artículo proporciona el marco metodológico fundamental:
\begin{itemize}
    \item Especificación con variable dependiente en logaritmos (consumo)
    \item Análisis de cointegración para relaciones de largo plazo
    \item Pruebas de raíces unitarias aplicadas a series macroeconómicas
    \item Interpretación de elasticidades consumo-remesas
\end{itemize}

\subsection{Literatura Complementaria de Soporte (Artículos 2010-2024)}

\textbf{1. Combes, J. L., \& Ebeke, C. (2011).} "Remittances and household consumption instability in developing countries". \textit{World Development}, 39(7), 1076-1089.\\
\textbf{DOI:} \url{https://doi.org/10.1016/j.worlddev.2010.10.006}
\begin{itemize}
    \item Metodología de cointegración aplicada a remesas y consumo
    \item Análisis de estabilidad en la relación de largo plazo
\end{itemize}

\textbf{2. Kumar, R. R. (2013).} "Remittances and economic growth: A study of Guyana". \textit{Economic Systems}, 37(3), 462-472.\\
\textbf{DOI:} \url{https://doi.org/10.1016/j.ecosys.2013.01.001}
\begin{itemize}
    \item Aplicación de pruebas KPSS y análisis de cointegración
    \item Especificación con variables de control macroeconómicas
\end{itemize}

\textbf{3. Sulemana, I., Anarfo, E. B., \& Quartey, P. (2019).} "International remittances and household food security in Sub-Saharan Africa". \textit{Migration and Development}, 8(2), 264-280.\\
\textbf{DOI:} \url{https://doi.org/10.1080/21632324.2018.1560926}
\begin{itemize}
    \item Análisis del impacto en consumo de hogares
    \item Metodología econométrica para series temporales
\end{itemize}

\textbf{4. Meyer, D., \& Shera, A. (2017).} "The impact of remittances on economic growth: An econometric model". \textit{EconomiA}, 18(2), 147-155.\\
\textbf{DOI:} \url{https://doi.org/10.1016/j.econ.2016.06.001}
\begin{itemize}
    \item Modelo de regresión con logaritmos
    \item Pruebas de diagnóstico completas
\end{itemize}

\textbf{5. Cazachevici, A., Havranek, T., \& Horvath, R. (2020).} "Remittances and economic growth: A meta-analysis". \textit{World Development}, 134, 105021.\\
\textbf{DOI:} \url{https://doi.org/10.1016/j.worlddev.2020.105021}
\begin{itemize}
    \item Meta-análisis de estudios de remesas
    \item Síntesis de metodologías econométricas efectivas
\end{itemize}

\textbf{6. Fromentin, V. (2017).} "The long-run and short-run impacts of remittances on financial development in developing countries". \textit{The Quarterly Review of Economics and Finance}, 66, 192-201.\\
\textbf{DOI:} \url{https://doi.org/10.1016/j.qref.2017.02.006}
\begin{itemize}
    \item Técnicas de cointegración de Johansen
    \item Análisis de largo plazo en economías en desarrollo
\end{itemize}

\section{Datos y Metodología Propuesta}

\subsection{Fuentes de Datos}
Los datos provienen de fuentes oficiales:
\begin{itemize}
    \item \textbf{Banco Central del Ecuador (BCE):} 
    \begin{itemize}
        \item Remesas recibidas (millones USD)
        \item Consumo final de los hogares (millones USD)
        \item PIB real (millones USD 2007)
        \item Índice de Precios al Consumidor
    \end{itemize}
    \item \textbf{Periodicidad:} Trimestral
    \item \textbf{Período:} 2015Q1 - 2024Q4 (40 observaciones)
\end{itemize}

\subsection{Variables del Modelo}

\begin{table}[H]
\centering
\caption{Definición de Variables}
\begin{adjustbox}{width=\textwidth}
\begin{tabular}{llll}
\toprule
\textbf{Variable} & \textbf{Notación} & \textbf{Descripción} & \textbf{Transformación} \\
\midrule
Dependiente & $logCONS_t$ & Consumo final de hogares & Logaritmo natural \\
Independiente principal & $logREM_t$ & Remesas recibidas & Logaritmo natural \\
Control 1 & $logPIB_t$ & PIB real & Logaritmo natural \\
Control 2 & $IPC_t$ & Índice de precios & Nivel \\
Control 3 & $logCRED_t$ & Crédito al sector privado & Logaritmo natural \\
\bottomrule
\end{tabular}
\end{adjustbox}
\end{table}

\subsection{Especificación del Modelo}
Siguiendo a Adams \& Cuecuecha (2013) y Kumar (2013), el modelo de largo plazo se especifica como:

\begin{equation}
logCONS_t = \beta_0 + \beta_1 logREM_t + \beta_2 logPIB_t + \beta_3 IPC_t + \beta_4 logCRED_t + \varepsilon_t
\end{equation}

Donde:
\begin{itemize}
    \item $\beta_1$ > 0: Elasticidad consumo-remesas (esperada entre 0.10-0.25)
    \item $\beta_2$ > 0: Efecto del ingreso nacional en el consumo
    \item $\beta_3$ < 0: Efecto de la inflación en el consumo real
    \item $\beta_4$ > 0: Efecto del crédito en el consumo
\end{itemize}

\subsection{Metodología Econométrica}

\subsubsection{Análisis de Integración (Paso 1-3)}
\begin{enumerate}
    \item \textbf{Identificación de componentes deterministas:}
    \begin{itemize}
        \item Regresión preliminar: $y_t = \alpha + \beta t + \varepsilon_t$
        \item Evaluar significancia de tendencia y constante
        \item Decisión sobre opción: trend, drift o noconstant
    \end{itemize}
    
    \item \textbf{Pruebas de raíz unitaria en niveles:}
    \begin{itemize}
        \item Test Augmented Dickey-Fuller (ADF)
        \item Test Phillips-Perron (PP)
        \item Test Kwiatkowski-Phillips-Schmidt-Shin (KPSS)
        \item Hipótesis: Series son I(1)
    \end{itemize}
    
    \item \textbf{Verificación en primeras diferencias:}
    \begin{itemize}
        \item Aplicar pruebas ADF, PP, KPSS a $\Delta y_t$
        \item Confirmar que las diferencias son I(0)
    \end{itemize}
\end{enumerate}

\subsubsection{Estimación y Diagnóstico (Paso 4-7)}
\begin{enumerate}
    \setcounter{enumi}{3}
    \item \textbf{Regresión MCO en niveles:}
    \begin{itemize}
        \item Estimación de la ecuación (1)
        \item Requerimiento mínimo: $R^2$ > 0.70
        \item Pruebas t para significancia individual ($\alpha$ = 0.05)
    \end{itemize}
    
    \item \textbf{Significancia de los Betas:}
    \begin{itemize}
        \item Test t-student para cada coeficiente
        \item Test F para significancia conjunta
        \item Interpretación económica de magnitudes
    \end{itemize}
    
    \item \textbf{Diagnóstico de problemas del modelo:}
    \begin{itemize}
        \item \textbf{Heterocedasticidad:} Test de Breusch-Pagan y White
        \item \textbf{Autocorrelación:} Test de Breusch-Godfrey LM y Durbin-Watson
        \item \textbf{Especificación:} Test RESET de Ramsey
        \item \textbf{Normalidad:} Test de Jarque-Bera y Shapiro-Wilk
        \item \textbf{Multicolinealidad:} Factor de Inflación de Varianza (VIF)
        \item \textbf{Estabilidad:} Tests CUSUM y CUSUMQ
    \end{itemize}
    
    \item \textbf{Predicción de residuos:}
    \begin{itemize}
        \item Obtener residuos estimados: $\hat{\varepsilon}_t$
        \item Análisis gráfico de residuos
        \item Correlograma de residuos
    \end{itemize}
\end{enumerate}

\subsubsection{Test de Cointegración (Paso 8)}
\begin{enumerate}
    \setcounter{enumi}{7}
    \item \textbf{Raíz unitaria de los residuos:}
    \begin{itemize}
        \item Test ADF sobre $\hat{\varepsilon}_t$ con opción \texttt{noconstant}
        \item Test PP sobre $\hat{\varepsilon}_t$ con opción \texttt{noconstant}
        \item Test KPSS sobre $\hat{\varepsilon}_t$ 
        \item Si residuos son I(0) → Cointegración confirmada
    \end{itemize}
\end{enumerate}

\section{Resultados Esperados}

\subsection{Hipótesis de Trabajo}
\begin{enumerate}
    \item \textbf{H1:} Las series $logCONS_t$ y $logREM_t$ son integradas de orden 1, I(1)
    \item \textbf{H2:} Existe una relación de cointegración entre consumo y remesas
    \item \textbf{H3:} La elasticidad consumo-remesas es positiva y significativa ($\beta_1 \in [0.10, 0.25]$)
    \item \textbf{H4:} Los residuos del modelo son estacionarios I(0), confirmando equilibrio de largo plazo
    \item \textbf{H5:} El modelo cumple con los supuestos clásicos de MCO
\end{enumerate}

\subsection{Implicaciones Económicas Esperadas}

Basándonos en la literatura internacional (Adams \& Cuecuecha, 2013; Kumar, 2013; Cazachevici et al., 2020):

\begin{itemize}
    \item \textbf{Elasticidad consumo-remesas:} Se espera que un incremento del 1\% en las remesas genere un aumento entre 0.10\% y 0.25\% en el consumo de los hogares, reflejando que parte de las remesas se destina al ahorro o inversión.
    
    \item \textbf{Efecto multiplicador:} Las remesas deberían mostrar un efecto multiplicador en la economía local a través del consumo, especialmente en bienes no transables.
    
    \item \textbf{Estabilidad de largo plazo:} La cointegración confirmaría que las remesas constituyen una fuente estable de financiamiento del consumo, validando políticas de facilitación de transferencias.
    
    \item \textbf{Complementariedad con el PIB:} Se espera que tanto las remesas como el PIB contribuyan positivamente al consumo, pero con diferentes elasticidades.
\end{itemize}

\section{Cronograma de Actividades}

\begin{table}[H]
\centering
\caption{Plan de Trabajo Detallado}
\begin{tabular}{lll}
\toprule
\textbf{Fase} & \textbf{Actividad} & \textbf{Días} \\
\midrule
1 & Recopilación de datos del BCE y validación & 2 \\
2 & Construcción de base de datos integrada & 1 \\
3 & Análisis descriptivo y gráficos preliminares & 2 \\
4 & Pruebas de raíces unitarias (ADF, PP, KPSS) & 2 \\
5 & Estimación del modelo de regresión MCO & 1 \\
6 & Diagnóstico completo de supuestos & 3 \\
7 & Pruebas de cointegración en residuos & 1 \\
8 & Interpretación económica de resultados & 2 \\
9 & Redacción del informe final & 3 \\
10 & Revisión, ajustes y formato APA & 1 \\
\midrule
& \textbf{Total} & \textbf{18 días} \\
\bottomrule
\end{tabular}
\end{table}

\section{Referencias Bibliográficas (Con DOI verificables)}

\begin{enumerate}
    \item Adams Jr, R. H., \& Cuecuecha, A. (2013). The impact of remittances on investment and poverty in Ghana. \textit{World Development}, 50, 24-40. DOI: \url{https://doi.org/10.1016/j.worlddev.2013.04.009}
    
    \item Banco Central del Ecuador. (2024). \textit{Información Estadística Mensual}. Recuperado de: \url{https://www.bce.fin.ec/informacioneconomica}
    
    \item Cazachevici, A., Havranek, T., \& Horvath, R. (2020). Remittances and economic growth: A meta-analysis. \textit{World Development}, 134, 105021. DOI: \url{https://doi.org/10.1016/j.worlddev.2020.105021}
    
    \item Combes, J. L., \& Ebeke, C. (2011). Remittances and household consumption instability in developing countries. \textit{World Development}, 39(7), 1076-1089. DOI: \url{https://doi.org/10.1016/j.worlddev.2010.10.006}
    
    \item Friedman, M. (1957). \textit{A theory of the consumption function}. Princeton University Press.
    
    \item Fromentin, V. (2017). The long-run and short-run impacts of remittances on financial development in developing countries. \textit{The Quarterly Review of Economics and Finance}, 66, 192-201. DOI: \url{https://doi.org/10.1016/j.qref.2017.02.006}
    
    \item Kumar, R. R. (2013). Remittances and economic growth: A study of Guyana. \textit{Economic Systems}, 37(3), 462-472. DOI: \url{https://doi.org/10.1016/j.ecosys.2013.01.001}
    
    \item Maridueña, Á. (2024). \textit{Manual de Cointegración en Stata: Caso aplicado PIB y Consumo}. Universidad Estatal de Milagro (UNEMI).
    
    \item Meyer, D., \& Shera, A. (2017). The impact of remittances on economic growth: An econometric model. \textit{EconomiA}, 18(2), 147-155. DOI: \url{https://doi.org/10.1016/j.econ.2016.06.001}
    
    \item Modigliani, F. (1966). The life cycle hypothesis of saving, the demand for wealth and the supply of capital. \textit{Social Research}, 33(2), 160-217.
    
    \item Sulemana, I., Anarfo, E. B., \& Quartey, P. (2019). International remittances and household food security in Sub-Saharan Africa. \textit{Migration and Development}, 8(2), 264-280. DOI: \url{https://doi.org/10.1080/21632324.2018.1560926}
    
    \item World Bank. (2024). \textit{Migration and Remittances Data}. Recuperado de: \url{https://www.worldbank.org/en/topic/migrationremittancesdiasporaissues/brief/migration-remittances-data}
\end{enumerate}

\section{Anexo: Código Stata Completo}

\begin{verbatim}
* ============================================
* ANÁLISIS DE COINTEGRACIÓN
* Remesas y Consumo de los Hogares en Ecuador
* Período: 2015Q1 - 2024Q4
* ============================================

clear all
set more off
cd "C:/Datos_Econometria"

* ============================================
* PREPARACIÓN DE DATOS
* ============================================

* Importar datos del BCE
import excel "datos_ecuador_bce.xlsx", sheet("Trimestral") firstrow clear

* Generar variable temporal
gen time = tq(2015q1) + _n - 1
format time %tq
tsset time

* Transformación logarítmica
gen logCONS = log(CONSUMO_HOGARES)
gen logREM = log(Remesas)
gen logPIB = log(PIB)
gen logCRED = log(CarteraBancoPrivado)

* Etiquetas
label var logCONS "Log Consumo Hogares"
label var logREM "Log Remesas"
label var logPIB "Log PIB Real"
label var IPC "Índice de Precios al Consumidor"
label var logCRED "Log Crédito Privado"

* ============================================
* PASO 1: IDENTIFICACIÓN DETERMINISTA
* ============================================

gen t = _n

* Para cada variable, evaluar tendencia
reg logCONS t
test t
reg logREM t
test t
reg logPIB t
test t

* ============================================
* PASO 2: PRUEBAS DE RAÍZ UNITARIA EN NIVELES
* ============================================

* Variable dependiente: logCONS
dfuller logCONS, trend
pperron logCONS, trend
kpss logCONS, trend

* Variable independiente principal: logREM
dfuller logREM, trend
pperron logREM, trend
kpss logREM, trend

* Variables de control
dfuller logPIB, trend
pperron logPIB, trend
kpss logPIB, trend

dfuller IPC, trend
pperron IPC, trend
kpss IPC, trend

dfuller logCRED, trend
pperron logCRED, trend
kpss logCRED, trend

* ============================================
* PASO 3: VERIFICACIÓN EN PRIMERAS DIFERENCIAS
* ============================================

* Generar primeras diferencias
gen d_logCONS = D.logCONS
gen d_logREM = D.logREM
gen d_logPIB = D.logPIB
gen d_IPC = D.IPC
gen d_logCRED = D.logCRED

* Evaluar componente determinista en diferencias
reg d_logCONS t
reg d_logREM t

* Tests en diferencias (usar drift si no hay tendencia)
dfuller d_logCONS, drift
pperron d_logCONS, drift
kpss d_logCONS, drift

dfuller d_logREM, drift
pperron d_logREM, drift
kpss d_logREM, drift

* ============================================
* PASO 4: REGRESIÓN MCO EN NIVELES
* ============================================

reg logCONS logREM logPIB IPC logCRED

* Guardar estadísticos clave
scalar R2 = e(r2)
scalar R2_adj = e(r2_a)
scalar F_stat = e(F)
scalar N_obs = e(N)

display "R-cuadrado: " R2
display "R-cuadrado ajustado: " R2_adj
display "Estadístico F: " F_stat
display "Observaciones: " N_obs

* ============================================
* PASO 5: SIGNIFICANCIA DE LOS BETAS
* ============================================

* La regresión anterior muestra t-stats automáticamente
* Test de significancia conjunta
test logREM logPIB IPC logCRED

* Test de restricciones específicas
test logREM = 0
test logPIB = 0

* ============================================
* PASO 6: DIAGNÓSTICO DE PROBLEMAS DEL MODELO
* ============================================

* 6.1 Heterocedasticidad
estat hettest
estat imtest, white

* 6.2 Autocorrelación
estat bgodfrey, lags(1/4)
estat dwatson

* 6.3 Especificación
estat ovtest

* 6.4 Normalidad
predict res1, resid
swilk res1
sktest res1
jb res1

* 6.5 Multicolinealidad
vif

* 6.6 Estabilidad estructural
estat sbcusum
cusum res1

* ============================================
* PASO 7: PREDICCIÓN DE RESIDUOS
* ============================================

predict ehat, resid

* Gráfico de residuos
tsline ehat, title("Residuos del Modelo") ///
    ytitle("Residuos") xtitle("Tiempo")

* Correlograma
ac ehat, lags(12)
pac ehat, lags(12)

* ============================================
* PASO 8: TEST DE COINTEGRACIÓN (RESIDUOS)
* ============================================

* Pruebas sobre residuos (sin constante)
dfuller ehat, noconstant
pperron ehat, noconstant
kpss ehat, noconstant

* Interpretación
display "========================================"
display "RESULTADOS DE COINTEGRACIÓN:"
display "Si los residuos son I(0), existe cointegración"
display "========================================"

* ============================================
* GRÁFICOS ADICIONALES
* ============================================

* Series en niveles
twoway (tsline logCONS, yaxis(1)) ///
       (tsline logREM, yaxis(2)), ///
       title("Consumo y Remesas en Logaritmos") ///
       legend(order(1 "Log Consumo" 2 "Log Remesas"))

* Relación bivariada
scatter logCONS logREM || lfit logCONS logREM, ///
    title("Relación Consumo-Remesas") ///
    xtitle("Log Remesas") ytitle("Log Consumo")

* ============================================
* TABLA RESUMEN DE RESULTADOS
* ============================================

quietly reg logCONS logREM logPIB IPC logCRED
outreg2 using "resultados_modelo.doc", replace word ///
    title("Modelo de Cointegración: Consumo y Remesas") ///
    addstat(R-squared, e(r2), F-stat, e(F), ///
            Durbin-Watson, r(dw))

log close
\end{verbatim}

\end{document}